\subsection{Soft margin}
\label{sec: svmLinearSoft}

So far, only perfectly separable cases were discussed, with the hyperplane which exactly separates two training sets (like presented on Figs. \ref{fig: knnSepTrainSetDef} and \ref{fig: knnSepTrainSetDis}). It is so called SVM with hard margin. It is clearly going to fail if training sets overlapped (like presented on Fig. \ref{fig: knnSepTrainSetOvr}). There is a simple extension to SVM method which allows some points to exists within the margin (between $\omega_0^* + \left<\vec\omega^*, \vec x\right> = 0$ and $\omega_0^* + \left<\vec\omega^*, \vec x\right> = \pm 1$ on Fig. \ref{fig: svmMargin}). It is so called SVM with soft margin. 

\begin{figure}
 \centering\input{figures/svmSoftMargin}
 \caption{The example of soft margin. {\it Note, it is just a demonstrative cartoon, not real calculations.}}
 \label{fig: svmSoftMargin}
\end{figure}

Lets associate the error $\varepsilon_i$ with every feature vector $\vec x_i$. The error defines how deep into the margin $\vec x_i$ falls:

\begin{itemize}
 \item[$\varepsilon_i = 0$] $\vec x_i$ is outside the margin
 \item[$\tau > \varepsilon_i > 0$] $\vec x_i$ is inside the margin but on the right side of the hyperplane
 \item[$\varepsilon_i > \tau$] $\vec x_i$ is on the wrong side of the hyperplane
\end{itemize}

as presented on Fig. \ref{fig: svmSoftMargin}. It requires to modify constraints defined by Eq. \ref{eq: optimization0C}:

\begin{equation}
  \forall_i\hspace{5pt} y_i \cdot \left(\omega_0 + \left<\vec\omega, \vec x_i\right>\right) + \varepsilon_i||\vec\omega||\geq \tau(\vec\omega, \omega_0)\cdot ||\vec\omega|| 
\end{equation}

After applying normalization bond $||\tau\cdot\vec\omega|| = 1$ the new constraints may be written as:

\begin{equation}
  \forall_i\hspace{5pt} y_i \cdot \left(\omega_0 + \left<\vec\omega, \vec x_i\right>\right) + \xi_i\geq 1
  \label{eq: softCsr}
\end{equation}

where $\xi_i \equiv \varepsilon_i||\vec\omega||$ is introduced for convenience. Once again the goal is to maximize $\tau$, so minimize $||\vec\omega||$, but also to minimize the errors $\xi_i$, with constraints defined by Eq. \ref{eq: softCsr}. In other words:

\begin{eqnarray}
 \text{minimize} & & Q(\vec\omega, \xi_1, ... , \xi_N) = \frac{1}{2}||\vec\omega||^2 + C\sum_{i=1}^N\xi_i \\ \nonumber
 \text{subject to} & & \forall_i\hspace{5pt} y_i \cdot \left(\omega_0 + \left<\vec\omega, \vec x_i\right>\right) + \xi_i\geq 1 \\ \nonumber
 \text{} & & \xi_i \geq 0
 \label{eq: optimizationSoft}
\end{eqnarray}

where $C > 0$ is a free parameter. Small $C$ makes the second term of $Q$ less important. It means the optimization is focused on finding large $\tau$ even for a price of having large errors $\xi_i$. On the other hand, large $C$ makes the optimization more sensitive to errors $\xi_i$ so small margin $\tau$ is preferred to minimize errors. It is demonstrated on Fig. \ref{fig: svmSoftMarginC}.

\begin{figure}
\hfill
\subfigure[Large $C$]{\begin{tikzpicture}[scale=0.7]

  \begin{axis}[xlabel = $x$, ylabel = $y$, xmin = -5.0, xmax = 5.0, ymin = -5.0, ymax = 5.0]
    
    \addplot[mark = *, , draw = none, color = morange] coordinates {(1,1)}; 
    \addplot[mark = *, , draw = none, color = morange] coordinates {(-1,2)}; 
    \addplot[mark = *, , draw = none, color = morange] coordinates {(-2,1)}; 

    \addplot[mark = *, , draw = none, color = mblue] coordinates {(-1,-2)}; 
    \addplot[mark = *, , draw = none, color = mblue] coordinates {(2,-2)}; 
    \addplot[mark = *, , draw = none, color = mblue] coordinates {(-3,-1)}; 

    \addplot[mark = *, , draw = none, color = mblue] coordinates {(0,-0.8)}; 

    \addplot[mark = *, , draw = none, color = morange] coordinates {(-1,-0.5)}; 
    
    \addplot[color = mred]         {-0.25*x - 0.5};
    \addplot[color = mred, dashed] {-0.25*x - 1};
    \addplot[color = mred, dashed] {-0.25*x + 0};   
    
  \end{axis}

\end{tikzpicture}}
\hfill
\subfigure[Small $C$]{\input{figures/svmSoftMarginCsmall}}
\hfill
\caption{The example of different choices of parameter $C$. {\it Note, it is just a demonstrative cartoon, not real calculations.}}
\label{fig: svmSoftMarginC}
\end{figure}

