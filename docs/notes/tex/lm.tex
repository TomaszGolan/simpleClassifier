\subsection{Lagrange multipliers}
\label{sec: lm}

Lagrange multipliers is the method to optimize (find minimum or maximum) of the function $f (\vec x) : \mathbb{R}^n \rightarrow \mathbb{R}$ with given constraint\footnote{The constraint could be given by $g (\vec x) = c$, but one can introduce $g'(\vec x) = g (\vec x) - c = 0$.} $g (\vec x) = 0$. The method is based on the fact that the gradient of the function $f$ must be parallel to the gradient of the constraint $g$:

\begin{equation}
 \nabla f (\vec x) = \lambda \nabla g (\vec x)
 \label{eq: lmBase}
\end{equation}

where $\lambda$ is a Lagrange multiplier. Why Eq. \ref{eq: lmBase} is true? Lets consider intuitive geometric explanation. If you want a strict proof grab a math book!

In general, the gradient $\nabla f$ gives the directions to head to optimize $f$. If going exactly in the direction of the gradient is impossible, but it is possible to go in the direction of a component of the gradient it still good (just increase / decrease is slower). If the gradient is perpendicular, there is no such component and the local optimum is found. Fig. \ref{fig: lmf} illustrates this.

\begin{figure}
\hfill
\subfigure[Looking of the local optimum of the function $f$.]{\scalebox{0.8}{\begin{tikzpicture}

  \begin{axis}[axis lines = none, axis equal]
    

    \addplot[mark = none, color = morange, thick, ->, >=latex] coordinates {(0.0, 1.0) (0.0, 1.25)};

    \addplot[mark = none, color = morange, thick, ->, >=latex] coordinates {(-0.5, 0.75) (-0.5, 1.0)};
    
    \addplot[smooth, color = mblack, thick, domain = -0.6:0.6] {-x*x + 1} node[above, pos = 0.75] {\color{mgray}$f$};
    
    \addplot[color = mgray, thick, dashed, domain = -0.35:-0.65] {x + 1.25};
    \addplot[color = mgray, thick, dashed, domain = -0.35:-0.65] {-x + 0.25};
    
    \addplot[mark = none, color = mblue, thick, ->, >=latex] coordinates {(-0.5, 0.75) (-0.4, 0.85)};

    \node(grad) [color = mgray] at (axis cs:-0.3,1.25) {\tiny gradients};
    
    \draw [color = mgray, ->, >=latex, shorten >= 2pt] (grad) -- (axis cs:0.0, 1.1);
    \draw [color = mgray, ->, >=latex, shorten >= 2pt] (grad) -- (axis cs:-0.5, 1.0);
    
    \node(move) [color = mgray] at (axis cs: 0.0, 0.75) {\tiny the direction of movement};
    
    \draw [color = mgray, ->, >=latex, shorten >= 2pt] (move.west) -- (axis cs:-0.45, 0.8);
    
    \node(nomove) [color = mgray, text width = 10cm, align = center] at (axis cs: 0.4, 1.2) {\tiny perpendicular gradient \\ no more moves \\};
    
    \draw [color = mgray, ->, >=latex, shorten >= 2pt] (nomove) -- (axis cs:0.0, 1.0);
    
  \end{axis}

\end{tikzpicture}} \label{fig: lmf}}
\hfill
\subfigure[Moving along the constraint $g$.]{\scalebox{0.8}{\begin{tikzpicture}

  \begin{axis}[axis lines = none, axis equal]
            
    \addplot[mark = none, color = mgreen, thick, ->, >=latex] coordinates {(-0.7, -0.343) (-0.9, -0.207)};
    
    \addplot[mark = none, color = morange, thick, ->, >=latex] coordinates {(-0.7, -0.343) (-0.8, 0.0)};

    \addplot[mark = none, color = morange, thick, ->, >=latex] coordinates {(0.5, 0.125) (0.3, 0.391)};    

    \addplot[mark = none, color = mgreen, thick, ->, >=latex] coordinates {(0.5, 0.125) (0.4, 0.258)};    
    
    \addplot[smooth, color = mblack, thick, domain = -1:1] {x*x*x} node[right, pos = 0.75] {\color{mgray}$g$};
    
    \addplot[mark = none, color = mblue, thick, ->, >=latex] coordinates {(-0.7, -0.343) (-0.6, -0.2)};
    
    \node(grad) [color = mgray] at (axis cs:-0.5,0.5) {\tiny gradients of $f$};

    \draw [color = mgray, ->, >=latex, shorten >= 2pt] (grad) -- (axis cs:-0.8,0);
    \draw [color = mgray, ->, >=latex, shorten >= 2pt] (grad) -- (axis cs:0.3, 0.391);
    
    \node(gradg) [color = mgray, text width = 3cm, align = center] at (axis cs:-0.3,-0.8) {\tiny gradients of $g$ \\ always perpendicular to $g$ \\};
    
    \draw [color = mgray, ->, >=latex, shorten >= 2pt] ([xshift=-1cm] gradg.north) -- (axis cs:-0.8, -0.3);
    
    \node(dir) [color = mgray, text width = 5cm, align = left] at (axis cs:0.4,-0.5) {\tiny the direction of movement \\ must be perpendicular to $\nabla g$ \\};
    
    \draw [color = mgray, ->, >=latex, shorten >= 2pt] (dir) -- (axis cs:-0.65,-0.3);
    
    \node(end) [color = mgray, text width = 3cm, align = center] at (axis cs:0.6,-0.2) {\tiny $\nabla f \parallel \nabla g$ \\ no more moves\\};
    
    \draw [color = mgray, ->, >=latex, shorten >= 2pt] (end) -- (axis cs:0.5,0.125);
    
  \end{axis}

\end{tikzpicture}} \label{fig: lmg}}
\hfill
\caption{The illustration of constraint optimization.}
\end{figure}

If constraint $g = 0$ is given, the movement is restricted to fulfill the condition. In other words, only moves along $g$ are possible. To keep $g$ constant, only moves perpendicular to the gradient of $g$ are allowed (otherwise increase or decrease of $g$ would occur). This is demonstrated on Fig. \ref{fig: lmg}. The direction of the movement is determined by $\nabla g$ (must be perpendicular) and $\nabla f$. More precisely it is given by a component of $\nabla f$ perpendicular to $\nabla g$. If $\nabla f$ is parallel to $\nabla g$ there is no such component and the local optimum (fulfilling the constraint) is found.

For convenience, lets introduce Lagrangian defined as:

\begin{equation}
 \mathcal{L} (\vec x, \lambda) = f (\vec x) - \lambda g (\vec x)
 \label{eq: lagrangian0}
\end{equation}

so the Eq. \ref{eq: lmBase} can be rewritten as:

\begin{equation}
 \nabla \mathcal{L} (\vec x, \lambda) = 0
 \label{eq: lagrEq}
\end{equation}

Lets go back to SVM optimization problem defined by Eq. \ref{eq: optimization}. The number of constraints there is equal to the size of the training set ($N$). Fortunately, the generalization of the Lagrangian (Eq. \ref{eq: lagrangian0}) is straightforward:

\begin{equation}
 \mathcal{L} (\vec x, \lambda_1, ..., \lambda_N) = f (\vec x) - \sum_{i = 1}^N\lambda_i g_i (\vec x)
 \label{eq: lagrangian}
\end{equation}

where $\lambda_i$ is Lagrange multiplier for the constraint $g_i$. The reasoning stays the same, but the movement of $f$ is now restricted by many constrains:

\begin{eqnarray*}
 \nabla f (\vec x) & = &  \lambda_1 \nabla g_1 (\vec x) \\
 \nabla f (\vec x) & = &  \lambda_2 \nabla g_2 (\vec x) \\
			 & ... & \\
 \nabla f (\vec x) & = &  \lambda_N \nabla g_N (\vec x)
\end{eqnarray*}

There is still one piece missing. The optimization problem given by Eq. \ref{eq: optimization} is constrained by inequalities. The generalization of Lagrange multipliers method to inequality constrains is given by Karush–Kuhn–Tucker (KKT) conditions, described in Sec. \ref{sec: kkt}.

