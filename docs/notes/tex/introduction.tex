\section{Introduction}
\label{sec: introduction}

A classifier, in machine learning, is an algorithm for finding a class of studying object, based on a set of objects with known class (training set). Thus, it is a type of supervised learning. 

If a classification can only distinguish between two training sets it is called binary classification, otherwise it is multiclass classification. SVM is a binary classifier, so several SVMs combination is required for multiclass classification. KNN is a multiclass classifier. In this note, only binary problems are considered, so combining binary classifiers problem is not discussed.

For any supervised learning the choice of a training set is crucial. They should uniformly cover the whole phase space. The optimal size of the training set is unique for a problem. It depends on the classification method, the complexity of the problem, and on the separability of classes. To determine the optimal size of the training set one can use learning curves, which presents the error of classification as a function of the training set size.

For complex problems, the choice of the proper features to feed a classifier is important and non-trivial. They must be chosen to well define a membership to some class. Unless you are a genius you will probably need to make some tests to determine the right set of properties to learn a classifier. In this note, only simple examples are discussed, so there is no problem with the choice of features.